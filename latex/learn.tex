\documentclass[a4paper,12pt]{article}
	\usepackage[UTF8]{ctex}
	\usepackage{color}
\begin{document}
	\title{My First Document}
	\author{Libro}
	\date{\todat}
	\maketitle
\section{I Section,Subsection,Paragraph,Subparagrah}
	content of section
	\subsection{subsection}
		content of subsection
	\paragraph{paragraph}
		content of paragraph
	\subparagraph{subparagraph}
		content of subparagraph
	\section{Type of the charactors}
	\subsection{text}
		\textit{words in italics}
		
		\textsl{words slanted}
		
		\textsc{words in smallcaps}
		
		\textbf{words in bold}
		
		\texttt{words in teletype}
		
		\textsf{sans serif words}
		
		\textrm{roman words}
		
		\underline{underlined words}
	
	\subsection{color}
		\colorbox{red}{fire}
		
		{\color{blue}ocean}
		
		nothing
		
		\colorbox{blue}{\color{red} the song of fire and ice}
	\subsection{size}
		normal size words
		
		{\tiny tiny words}
		
		{\scriptsize scriptsize words}
		
		{\footnotesize footnotesize words}
		
		{\small small words}
		
		{\large large words}
		
		{\Large Large words}
		
		{\LARGE LARGE words}
		
		{\huge huge words}
	\section{list}
			\begin{enumerate}
			\item First thing
			\item Second thing
			\begin{itemize}
			\item A sub-thing
			\item Another sub-thing
			\end{itemize}
			\item Third thing
			\end{enumerate}

1. x \mod a : 这个命令将在 x 和 mod 之间产生很大的间隔.

x   mod a

2. x \bmod a : 这个命令在 x 和 mod 之间产生的间隔比较合适.

x mod a

3. x \pmod a : 这个命令将在上面的基础上添加小括号.

x (mod a)

4. x \pod a : 省略 mod .

x (a)
 

例如: x\equiv y \pmod p 得到


\end{document}


other commands:
 \data{November 2013